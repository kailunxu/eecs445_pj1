\documentclass[11pt, answers]{exam}
\usepackage[margin=1in]{geometry}
\usepackage{amsfonts, amsmath, amssymb, amsthm}
\usepackage{mathtools}
\usepackage{enumerate}
\usepackage{listings}
\usepackage{hyperref}
\usepackage[boxed]{algorithm}
\usepackage[noend]{algpseudocode}
\usepackage{tikz}
\usepackage{float}

%
% Basic Document Settings
%

\topmargin=-0.45in
\evensidemargin=0in
\oddsidemargin=0in
\textwidth=6.5in
\textheight=9.0in
\headsep=0.25in

\linespread{1.1}

\pagestyle{headandfoot}
\lhead{\hmwkAuthorName\ (\hmwkUniqname)}
\rhead{\hmwkClass\ : \hmwkType\ \#\hmwkNumber\ (Due \hmwkDue)}
\cfoot{\thepage}
% \renewcommand\headrulewidth{0.4pt}
% \renewcommand\footrulewidth{0.4pt}

\setlength\parindent{0pt}

%
% Create Problem Sections
%
\qformat{\hfill}

\newcommand{\hmwkType}{Project}
\newcommand{\hmwkNumber}{1}
\newcommand{\hmwkClass}{EECS 445}
\newcommand{\hmwkDue}{Tuesday, February 11th at 11:59pm}
\newcommand{\hmwkAuthorName}{\textbf{Your Name}}
\newcommand{\hmwkUniqname}{\textbf{uniqname}}


%
% Title Page
%

\author{\hmwkAuthorName}

%
% Various Helper Commands
%

% space of real numbers \R
\newcommand{\R}{\mathbb{R}}

% expected value \EX
\DeclareMathOperator{\EX}{\mathbb{E}}

% For partial derivatives \pderiv{}{}
\newcommand{\pderiv}[2]{\frac{\partial}{\partial #1} (#2)}

% argmax \argmax
\DeclareMathOperator*{\argmax}{arg\,max}

% sign \sign
\DeclareMathOperator{\sign}{sign}

% norm \norm{}
\DeclarePairedDelimiter{\norm}{\lVert}{\rVert}

% Keys
\newcommand{\key}[1]{\fbox{{\sc #1}}}
\newcommand{\ctrl}{\key{ctrl}--}
\newcommand{\shift}{\key{shift}--}
\newcommand{\run}{\key{run} \ }
\newcommand{\runkey}[1]{\run \key{#1}}
\newcommand{\extend}{\key{extend} \ }
\newcommand{\kkey}[1]{\key{k$_{#1}$}}

\begin{document}
  \section{Introduction}%
  \label{sec:introduction}

  \section{Feature Extraction}%
  \label{sec:feature_extraction}

  \begin{solution}\begin{parts}
    \part
    \part
    \part
  \end{parts}\end{solution}

  \section{Hyperparameter and Model Selection}%
  \label{sec:hyperparameter_and_model_selection}

  \subsection{Hyperparameter Selection for a Linear-Kernel SVM}%
  \label{sub:hyperparameter_selection_for_a_linear_kernel_svm}

  \begin{solution}\begin{parts}
    \part
    \part
    \part
    \part
    \part
    \part
    \part
  \end{parts}\end{solution}

  \subsection{Hyperparameter Selection for a Quadratic-Kernel SVM}%
  \label{sub:hyperparameter_selection_for_a_quadratic_kernel_svm}

  \begin{solution}\begin{parts}
    \part \begin{enumerate}[(i)]
      \item
      \item
    \end{enumerate}
    \part
  \end{parts}\end{solution}

  \subsection{Learning Non-Linear Classifiers with a Linear-Kernel SVM}%
  \label{sub:learning_non_linear_classifiers_with_a_linear_kernel_svm}

  \begin{solution}\begin{parts}
    \part
    \part
  \end{parts}\end{solution}

  \subsection{Linear-Kernel SVM with L1 Penalty and Squared Hinge Loss}%
  \label{sub:linear_kernel_svm_with_l1_penalty_and_squared_hinge_loss}

  \begin{solution}\begin{parts}
    \part
    \part
    \part
    \part
  \end{parts}\end{solution}

  \section{Asymmetric Cost Functions and Class Imbalance}%
  \label{sec:asymmetric_cost_functions_and_class_imbalance}

  \subsection{Arbitrary Class Weights}%
  \label{sub:arbitrary_class_weights}

  \begin{solution}\begin{parts}
    \part
    \part
    \part
  \end{parts}\end{solution}

  \subsection{Imbalanced data}%
  \label{sub:imbalanced_data}

  \begin{solution}\begin{parts}
    \part
    \part
  \end{parts}\end{solution}

  \subsection{Choosing appropriate class weights}%
  \label{sub:choosing_appropriate_class_weights}

  \begin{solution}\begin{parts}
    \part
    \part
  \end{parts}\end{solution}

  \subsection{The ROC curve}%
  \label{sub:the_roc_curve}

  \begin{solution}
  \end{solution}

  \section{Challenge}%
  \label{sec:challenge}

  \begin{solution}
  \end{solution}
\end{document}
